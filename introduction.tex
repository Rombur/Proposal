\section{Introduction}
The transport of photons and electrons has many applications in medical
physics; particularly in radiotherapy. Radiotherapy uses photons and
charged particles to damage the DNA of cancerous cells. When using photons in
this process, free electrons are created that ionizes the environment and creates 
free radicals that damage the cells. The absorbed dose, defined as the energy 
deposited per unit of mass, is used to gauge whether a cell will be destroyed
by the radiation. Several methods can be applied to compute the dose distribution
in the body: semi-analytic, deterministic and Monte-Carlo
methods. Monte-Carlo methods yield very accurate results, however they are
slow to converge and remain too slow for effective clinical use
\cite{acuros}. Other methods, such as pencil-beam convolution and
convolution-superposition, employ pre-calculated Monte-Carlo dose kernels,
which are then locally scaled to approximate photon and electron transport in
the presence of heterogeneities. These methods present some issues in the
presence of large density gradients, like those found at interfaces between
different materials: air, bone, lung and soft tissue
\cite{acuros,seco,krieger}. If enough cells are used, deterministic methods
can be accurate even on interfaces between materials \cite{acuros}. These
methods are faster than Monte-Carlo methods but slower than the semi-analytic
methods.\\
The main purpose of the research proposed is to optimize the dose distribution
while using a deterministic method to compute the dose. One difficulty of this
calculation comes from the transport of the electrons. The reason for this is
because the electrons
are charged particles and have very anisotropic scattering due
to their Coulomb interactions with other particles. This
anisotropy causes some complications since the standard Legendre expansion
representing the cross sections would require hundreds of terms.  A common
approximation uses a Dirac distribution to model the forward-peaked
scattering of the electrons and a continuous slowing-down for energy loss due
to Coulomb interactions. This allows the Legendre expansion of the cross
section to be kept to a low order. Thus, we obtain an equation that is the
transport equation for neutral particle with an additional term for the
continuous slowing-down of the electrons and where the scattering
cross-section contains a Dirac distribution.\\
In this research, we will focus on developing a deterministic method which 
can compute the dose quickly and without using too much memory. To 
decrease the time needed to compute the dose, we will parallelized the program 
and we will use an acceleration scheme. The parallelization of the program will 
be done using the existing capabilities of the Parallel Deterministic Transport 
(PDT) code developed at Texas A\&M University. PDT has the ability to solve the 
transport equation for neutral particle and we will use this existing framework to 
add the electron transport. In addition to parallelization, a good preconditioner 
will also speed-up the process \cite{adams}. The most common preconditioners for the
transport equation are the P1SA and the DSA methods. These methods have been shown
to be efficient when the scattering anisotropy is not too large ($\mu < 0.5$). 
However, for electron transport the scattering anisotropy is very large and these 
methods have to be modified \cite{russe,multisweep}. In this research, we will
test the capability of a new P1SA method even when we know that this preconditioner
will not be stable for large anisotropy \cite{multisweep}. Therefore some
modifications of the standard way to apply the scheme will have to be made.

Since the photons are sent using beams,
ray effect can be very strong and therefore a first collision source will be
implemented. The problem with the existing algorithms is that for the
ray-tracing they either need that every processor knows the entire mesh which
requires a lot of memory or they require the processors that contains the part
of the mesh with the sources
to make the most calculations which is no scalable either. This problem has
been studied extensively in others disciplines \cite{progressive, data} but much less 
in nuclear engineering even if the ray-tracing using GPU is studied a lot for
medical applications \cite{gpu}. Finally, we will couple the transport code to
an optimization code to compute the best way to send the photons beams. The 
difficulty of this calculation resides in the local minima which can be large.
