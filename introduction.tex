\section{Introduction}
The transport of photons and electrons has many applications in medical
physics and particularly in radiotherapy. Radiotherapy uses photons and
charged particles to damage the DNA of cancerous cells, While using photons,
free electrons are created; this ionizes the environment and creates free
radicals that damage the cells. One quantity used to gauge whether a cell will
die due to radiation is the absorbed dose, defined as the energy deposited per
unit of mass. Several methods can be applied to compute the dose distribution
in the body: semi-analytic methods, deterministic methods and Monte-Carlo
methods. Monte-Carlo methods yield very accurate results, however they are
slow to converge and, thus, remain too slow for effective clinical use
\cite{acuros}. Other methods, such as pencil-beam convolution and
convolution-superposition, employ pre-calculated Monte-Carlo dose kernels,
which are then locally scaled to approximate photon and electron transport in
the presence of heterogeneities. These methods present some issues in the
presence of large density gradients, such as those found at interfaces between
different materials: air, bone, lung and soft tissue
\cite{acuros,seco,krieger}. If enough cells are used, deterministic methods
can be accurate even on interfaces between materials \cite{acuros}. These
methods are faster than Monte-Carlo methods but slower than the semi-analytic
methods.\\
The main purpose of the research proposed is to use deterministic methods to
use deterministic method to compute the dose. The difficulty of this
calculation comes from the transport of the electrons. Because the electrons
are charged particles, they have very anisotropic scattering scattering due
to their interactions with other particles through Coulomb interaction. This
anisotropy causes some complications since the standard Legendre expansion
representing the cross sections would require hundreds of terms.  A common
approximation is to use a Dirac distribution to model the forward-peaked
scattering of the electrons and a continuous slowing down for energy loss due
to Coulomb interactions. This allows the Legendre expansion of the cross
section to be kept to a low order. However, an exact integration of the Dirac
distribution requires to use Galerkin quadratures \cite{graal} that demand a
significant amount of memory. These quadratures need the number of flux
moments and the number of angular fluxes to be equal and this number varies as
$\frac{n(n+2)}{8}$ per octant with the order of the $S_n$ method. For medical
applications, we need a method which is also fast. In this research, we will
focus on developing a deterministic method which does not require too much
memory, to be able to compute the dose quickly and without using too much
memory. To decrease the time needed to compute the dose, we will parallelized
the program and we will use an acceleration scheme. We will try several
methods to decrease the memory need, which is a limiting factor to increase
the number of cells and therefore the accuracy of the solution. Finally, we
will couple the transport code to an optimization code to compute the best way
to send the photons beams. 
