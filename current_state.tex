\section{Current state of the problem}
There are two parts of the problem. First, we need to solve the transport
equation for the photons and the electrons and then, we need to solve the
optimization problem.
\subsection{Transport equation for the photons and the electrons}
The equation that we want to solve is the Boltzmann-CSD equation (the
variables are omitted for \hbox{brevity) :}
\begin{equation}
\begin{split}
\bo \cdot \bn \Psi + \Sigma_t \Psi = \int_{4\pi}\int_0^{\infty}
\Sigma_s\(\bo\cdot\bo', E'\rightarrow E\)\Psi\(\bo',E'\)dE' d\bo'+
\frac{\partial S \Psi}{\partial E} +Q
\label{b-csd}
\end{split}
\end{equation}
where :
\begin{itemize}
\item $\Psi$ is the angular flux
\item $\Sigma_t$ is the smooth-component of the total macroscopic cross
section
\item $\Sigma_s$ is the smooth-component of the macroscopic differential
scattering cross section
\item $Q$ is a volumetric source
\item $\bo = (\mu,\phi)$
\item $\mu$ is the cosine of the directional polar angle
\item $\phi$ is the directional azimuthal angle
\item $S$ is the restricted stopping power :
\begin{equation}
S(E) = 2\pi \int_0^E \int_{-1}^1 \Sigma_{ss} (E\rightarrow E',\mu_0)
\(E-E'\)d\mu_0 dE'
\end{equation}
\item $\mu_0 = \mu' \mu + \sqrt{\(1-\mu'^2\)\(1-\mu^2\)} \cos\(\phi'-\phi\)$
\item $\Sigma_{ss}\(E\rightarrow E', \mu_0\)$ denotes the forward-peaked
scattering cross section.
\end{itemize}
Standard boundary conditions can be applied to (\ref{b-csd}), the most likely
being the vacuum boundary conditions :
\begin{equation}
\Psi(\br,\bo,E) = 0\ \textrm{ for } \bo \cdot \bs{n} < 0 \ \textrm { and } \br
\in \partial \mathcal{D}_v
\end{equation}
and the incoming flux boundary conditions :
\begin{equation}
\Psi(\br,\bo,E) = g(\br,\bo,E)\ \textrm{ for } \bo \cdot \bs{n} < 0 \ \textrm { and } \br \in \partial \mathcal{D}_i
\end{equation}
where $\partial \mathcal{D}_v$ is the boundary of the domain where vacuum
conditions are applied and $\partial \mathcal{D}_i$ is the boundary of the
domain where incoming flux conditions are applied.\\
The state-of-art for deterministic electron-photon calculation is the code
Acuros\cite{acuros}. It employs linear discontinuous finite-element on
unstructured mesh consisting of tetrahedral elements and an adaptive mesh
refinement method. It uses a discrete ordinates for the angular discretization 
and compute the first
collision source by computing an analytical uncollided flux. The
energy discretization is done with standard multigroup method and the
energy derivative of the CSD operator is discretized using linear
discontinuous finite-element.

\subsection{Optimization problem}
The modern radiotherapy uses the Intensity Modulated Radiation Therapy (IMRT) as 
one of the methods to treat cancer. IMRT allows several beams with 
different intensity profiles and the goal is to deliver a sufficient dose to the 
tumor while sparing healthy organs. To optimize the intensity profile, it is very 
common to divide the beams in small beamlets, each of them having a constant 
intensity. In a real application, the number of beamlets is around a thousand. 

This problem is very complex and a lot of objective functions and constraints 
\cite{math,complexity,minima,dose-volume} have been 
proposed. Moreover, there are a lot of methods \cite{dose-volume} used to solve 
these optimization problems such as Monte-Carlo method with steepest descent
\cite{complexity}, linear programming, filtered back-projection, simulated
annealing, dynamically penalized likelihood, projections onto convex, active
set method and simulated dynamics\cite{dose-volume}.

The most used objective function is :
\begin{equation}
\min  \sum_{(i,j,k)\in \mathcal{T}} \(D_{ijk}-\delta_{ijk}\)^2 
\label{objective}
\end{equation}
where $\delta_{ij}$ is the prescribed dose for the voxel $ijk$, $D_{ijk}$ is the
actual dose for the voxel and $\mathcal{T}$ is the tumor. 

This objective function tries to to minimize the difference between the planned 
dose and the dose in the tumor. The constraints used are dose-volume constraints
\cite{complexity,minima,dose-volume}. The idea is that we do not want to have more 
than a given percentage of the volume of an organ to receive more than a given dose. 
Mathematically, we have for each constraint :
\begin{equation} 
\sum_{ijk \in \mathcal{D}} \nu\(D_{ijk}-\delta_{ijk}\)\frac{V_{ijk}}{V} \leq \gamma
\label{constraint}
\end{equation}
where $V_{ij}$ is the volume of voxel $ij$, $V$ is the total volume of the organ, 
$\nu(x)$ is the Heaviside function, $\gamma$ is the tolerance percentage
volume and $\mathcal{D}$ is the domain studied. We also impose the dose to be 
positive or zero :
\begin{equation}
D_{ij} \geq 0
\label{constraint2}
\end{equation}

It is well known that even if the objective function is convex, the problem has 
several local minima \cite{minima}. The reason is that because the domain is not 
convex, the number of local minima can increase factorially with the number of 
beamlets. Since we can have more than a thousand beamlets, the 
number of local minima can be large. To avoid the local minima problem, 
we can rewrite the optimization problem as a mixed-integer linear problem. 
This does not have local minima but the calculation time can be prohibitive 
\cite{minima}.  

